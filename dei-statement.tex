\documentclass{article}
\usepackage[margin=1in]{geometry} % Optional: Adjust page margins
\usepackage{fancyhdr} % For custom headers
\usepackage{booktabs}
\usepackage{hyperref}
\usepackage{xcolor}
\usepackage{enumitem}
\usepackage{kotex}
\usepackage{titlesec}
\usepackage{parskip}

% make reference blue
\hypersetup{
  colorlinks,
  linkcolor={blue},
  citecolor={blue},
  urlcolor={blue}
}
\renewcommand{\normalsize}{\fontsize{10}{12pt}\selectfont}
% \titleformat{\section}
%   {\normalfont\fontsize{13}{13}\bfseries}{\thesection}{1em}{}

% Define the global footer
\pagestyle{fancy}
\fancyhf{} % Clear all header/footer fields
\fancyfoot[R]{Page \thepage} % Footer centered on all pages


% Define the header for the first page
\fancypagestyle{firstpage}{
    \fancyhf{} % Clear all header/footer fields
    % \fancyhead[L]{\textbf{DEI Statement}} % Left-aligned header
    \fancyhead[L]{\textbf{Statement on Commitment to Diversity}} % Left-aligned header
    % \fancyhead[C]{Center Header (optional)} % Centered header
    \fancyhead[R]{\textbf{Seongmin Lee}} % Right-aligned header
    \fancyfoot[R]{Page \thepage} % Footer right
}

\begin{document}

\thispagestyle{firstpage} % Apply the custom header only to the first page

\subsubsection*{Foundations of My Commitment to Diversity, Equity, and Inclusion}

Growing up, I often found myself in environments that highlighted the importance of diversity, equity, and inclusion (DEI). Much of my early education took place in settings with significant gender imbalances. I attended a single-sex middle school in South Korea and later enrolled in Ulsan Science High School and the Korea Advanced Institute of Science and Technology (KAIST), both specializing in science and engineering. These environments revealed the systemic challenges faced by women, underscoring the need for deliberate efforts to foster equity. At the same time, I was fortunate to work in research groups that actively prioritized DEI. During my internships and graduate studies, I had advisors who created inclusive spaces—one of whom was a woman, and another who organized annual Ada Lovelace Day events celebrating women in STEM. Participating in these initiatives showed me how fostering inclusivity supports individual well-being and enhances team collaboration and innovation. These experiences have shaped my commitment to DEI in both academic and professional contexts.

Building on these foundations, my understanding of DEI evolved further during my time in Europe. As an Asian male navigating new cultural and academic systems, I experienced firsthand the challenges of being part of a minority group. The Max Planck Institute's international and welcoming community made this transition not only manageable but deeply enriching. Collaborating with colleagues from diverse backgrounds broadened my understanding of DEI, reaffirming its role not only as a moral responsibility but also as a cornerstone of innovation and excellence. However, I recognize that achieving the full benefits of diversity requires deliberate action—such as \textbf{promoting diversity through recruitment} and \textbf{fostering inclusive environments}.

\subsubsection*{Promoting Diversity through Recruitment} 

My understanding of DEI deepened through leadership roles in extracurricular activities. As the club master of the undergraduate Classical Guitar Club and a managing board member of the graduate Badminton Club—both of which were among the larger clubs at my institutes, with 30 to 50 active members—I prioritized fostering diversity during recruitment. While inclusiveness and a welcoming attitude were central values, I recognized that achieving diversity within a team is often the first step toward building an inclusive community. For individuals from minority groups, having peers who genuinely understand and share their experiences is especially important.

At KAIST, I worked to encourage a balanced and diverse membership in both clubs, achieving a roughly 50\% gender balance. This approach not only promoted diversity but also enabled us to create a more supportive environment. For example, we organized small peer support groups for female members, providing spaces for mutual understanding, support, and joy in both club activities and broader academic life. These initiatives fostered active participation from all members and ensured continuity of these values—successive boards included women who carried forward the club's commitment to diversity, equity, and inclusion.

These experiences taught me the importance of thoughtful recruitment as a means to promote diversity, as well as the sustained support needed to build inclusive communities. As a faculty member, I will apply these lessons in recruiting diverse teams, mentoring students, and fostering a culture where everyone feels valued and empowered. By aligning these efforts with the institution's broader commitment to excellence and inclusion, I aim to contribute to an environment where DEI principles are not just ideals but lived experiences.

\subsubsection*{Fostering Inclusive Environments}

Creating welcoming environments where individuals from all backgrounds feel supported has been a key focus of my efforts. Currently, I serve on the Early Career Researchers (ECR) Board in the Cluster of Excellence CASA, where we oversee funding allocations and organize events such as summer schools, symposiums, and regular meetings. Addressing inclusiveness is central to our work, and we regularly incorporate feedback to ensure our programs meet the needs of a diverse community. For instance, I proposed “speed-dating” activities to connect German-speaking and non-German-speaking members, fostering engagement and helping international researchers feel more integrated into the community.

I also take personal responsibility for creating a welcoming atmosphere. As an international researcher, I understand the importance of having someone reach out during transitions into a new community. I make a point to welcome newcomers—whether in my research group or elsewhere—by initiating conversations, encouraging participation in activities, and fostering a friendly environment. One intern remarked, “Seongmin's welcoming attitude made a big difference. He was the first to reach out, and he created an atmosphere where everyone felt comfortable and included.”

My earlier leadership experiences in the Classical Guitar and Badminton Clubs also shaped my approach to inclusiveness. I organized small, diverse support groups pairing seniors with juniors to foster mentorship and camaraderie, helping members navigate both club activities and broader academic life. Drawing from my experiences of being both in the majority and minority, I understand how inclusiveness enhances individual well-being and strengthens community cohesion.

I will promote diversity through recruitment within my research group and foster welcoming environments across my research group, department, and institute. By thoughtfully recruiting diverse teams and establishing support systems that ensure individuals feel valued and empowered, I will cultivate a culture where collaboration and innovation thrive. These approaches will enhance individual well-being while contributing to stronger, more inclusive academic communities that reflect the values of diversity, equity, and inclusion.

\end{document}