\documentclass[letterpaper,11pt]{article}

\usepackage{latexsym}
\usepackage[empty]{fullpage}
\usepackage{titlesec}
\usepackage{marvosym}
\usepackage[usenames,dvipsnames]{color}
\usepackage{verbatim}
\usepackage{enumitem}
\usepackage{fancyhdr}
\usepackage{lastpage}
\usepackage[english]{babel}
\usepackage{tabularx}
\usepackage{fontawesome5}
\usepackage{multicol}

% \usepackage{showframe}

\usepackage{bibentry}
\makeatletter\let\saved@bibitem\@bibitem\makeatother
\usepackage[hidelinks]{hyperref}
\makeatletter\let\@bibitem\saved@bibitem\makeatother

\setlength{\multicolsep}{-3.0pt}
\setlength{\columnsep}{-1pt}
\input{glyphtounicode}


%----------FONT OPTIONS----------
% sans-serif
% \usepackage[sfdefault]{FiraSans}
% \usepackage[sfdefault]{roboto}
% \usepackage[sfdefault]{noto-sans}
% \usepackage[default]{sourcesanspro}

% serif
% \usepackage{CormorantGaramond}
% \usepackage{charter}


\pagestyle{fancy}
\fancyhf{} % clear all header and footer fields
\fancyfoot{}
\fancyfootoffset[R]{0in}
% \fancyfoot[R]{\thepage/\pageref{LastPage}}
\rfoot{\thepage/\pageref{LastPage}}
\renewcommand{\headrulewidth}{0pt}
\renewcommand{\footrulewidth}{0pt}

% Adjust margins
\addtolength{\oddsidemargin}{-0.6in}
\addtolength{\evensidemargin}{-0.5in}
\addtolength{\textwidth}{1.19in}
\addtolength{\topmargin}{-.4in}
\addtolength{\textheight}{.8in}

\urlstyle{same}

\raggedbottom
\raggedright
\setlength{\tabcolsep}{0in}

% Sections formatting
\titleformat{\section}{
  \vspace{-4pt}\scshape\raggedright\large\bfseries
}{}{0em}{}[\color{black}\titlerule \vspace{-5pt}]

% Ensure that generate pdf is machine readable/ATS parsable
\pdfgentounicode=1

%-------------------------
% Custom commands
\newcommand{\resumeItem}[1]{
  \item\small{
    {#1 \vspace{-2pt}}
  }
}

\newcommand{\classesList}[4]{
    \item\small{
        {#1 #2 #3 #4 \vspace{-2pt}}
  }
}

\newcommand{\resumeSubheading}[4]{
  \vspace{-2pt}\item
    \begin{tabular*}{1.0\textwidth}[t]{l@{\extracolsep{\fill}}r}
      \textbf{#1} & \textbf{\small #2} \\
      \textit{\small#3} & \textit{\small #4} \\
    \end{tabular*}\vspace{-7pt}
}

\newcommand{\resumeSubSubheading}[2]{
    \item
    \begin{tabular*}{0.97\textwidth}{l@{\extracolsep{\fill}}r}
      \textit{\small#1} & \textit{\small #2} \\
    \end{tabular*}\vspace{-7pt}
}

\newcommand{\resumeProjectHeading}[2]{
    \item
    \begin{tabular*}{1.001\textwidth}{l@{\extracolsep{\fill}}r}
      \small#1 & \textbf{\small #2}\\
    \end{tabular*}\vspace{-7pt}
}

\newcommand{\resumeSubItem}[1]{\resumeItem{#1}\vspace{-4pt}}

\renewcommand\labelitemi{$\vcenter{\hbox{\tiny$\bullet$}}$}
\renewcommand\labelitemii{$\vcenter{\hbox{\tiny$\bullet$}}$}

\newcommand{\resumeSubHeadingListStart}{\begin{itemize}[leftmargin=0.0in, label={}]}
\newcommand{\resumeSubHeadingListEnd}{\end{itemize}}
\newcommand{\resumeItemListStart}{\begin{itemize}}
\newcommand{\resumeItemListEnd}{\end{itemize}\vspace{-5pt}}



%-------------------------------------------
%%%%%%  RESUME STARTS HERE  %%%%%%%%%%%%%%%%%%%%%%%%%%%%


\begin{document}

\bibliographystyle{unsrt}
\nobibliography{ref}

%----------HEADING----------
% \begin{tabular*}{\textwidth}{l@{\extracolsep{\fill}}r}
%   \textbf{\href{http://sourabhbajaj.com/}{\Large Sourabh Bajaj}} & Email : \href{mailto:sourabh@sourabhbajaj.com}{sourabh@sourabhbajaj.com}\\
%   \href{http://sourabhbajaj.com/}{http://www.sourabhbajaj.com} & Mobile : +1-123-456-7890 \\
% \end{tabular*}

% \begin{center}
{\Huge \scshape Seongmin Lee} \\ \vspace{10pt}
Max Planck Institute for Security and Privacy (MPI-SP) \\
Universitätsstraße 140 \\
44799 Bochum \\
Germany \\ \vspace{5pt}
\small \raisebox{-0.1\height}\faPhone\ +49 177 783 5480  ~ \href{mailto:seongmin.lee@mpi-sp.org}{\raisebox{-0.2\height}\faEnvelope\  \underline{seongmin.lee@mpi-sp.org}} ~
\href{https://scholar.google.com/citations?user=-YSnc6kAAAAJ&hl=en}{\raisebox{-0.2\height}\faUniversity\ \underline{Google Scholar}}  ~
% \href{https://github.com/}{\raisebox{-0.2\height}\faGithub\ \underline{github.com/username}}
% \vspace{-8pt}
% \end{center}

\section{Research Interests}

My research interest lies in dynamic program analysis, especially using statistical methods on dynamic information from execution to reason about a program's semantic properties.
The goal of my research is to bring program analysis closer to real-world circumstances regarding the scale and complexity of software within the presence of non-experimental or missing data in the analysis.


%-----------EDUCATION-----------
\section{Education and Employment}
\resumeSubHeadingListStart
% \resumeSubheading
%   {Korea Advanced Institue of Science and Technology (KAIST)}{Sep. 2017 -- May 2021}
%   {Bachelor of Science in Computer Science}{City, State}
\vspace{-2pt}
\item Max Planck Institute for Security and Privacy \hfill Germany \vspace{-2pt}
\vspace{-2pt} \item \hspace{10pt} \textbf{Postdoctoral Researcher, Software Security Research group} \hfill  Sep. 2022 -- Present
\vspace{-5pt} \item \hspace{20pt} Group head: Dr. Marcel Böhme

\vspace{-2pt}
\item Korea Advanced Institute of Science and Technology \hfill Republic of Korea \vspace{-2pt}
\vspace{-2pt} \item \hspace{10pt} \textbf{Doctor of Philosophy, School of Computing} \hfill  Sep. 2016 -- Aug. 2022
\vspace{-5pt} \item \hspace{20pt} Advisor: Dr. Shin Yoo
% \vspace{-5pt} \item \hspace{20pt} GPA - 4.02/4.3

\vspace{-2pt} \item \hspace{10pt} \textbf{Bachelor of Science, School of Computing} \hfill  Feb. 2012 -- Aug. 2016
\vspace{-5pt} \item \hspace{10pt} \textbf{Bachelor of Science, Department of Mathematical Sciences}
% \vspace{-5pt} \item \hspace{20pt} GPA - 3.48/4.3
\resumeSubHeadingListEnd

\section{Publications}

\textbf{Preprints}\vspace{-4pt}
\begin{itemize}
  \item \bibentry{lee2024unseen} -- \emph{Under review}
  \item \bibentry{lee2024chomi} -- \emph{Under review}
  \item \bibentry{Lee:2021ul}
\end{itemize}


\textbf{Peer-Reviewed Journal Articles}\vspace{-4pt}

\begin{itemize}
  \item \bibentry{Lee:2021ua}
  \item \bibentry{Lee:2020ab}
\end{itemize}

\textbf{Peer-Reviewed Conference and Workshop Papers}\vspace{-4pt}

\begin{itemize}
  \item \bibentry{Liyanage:2024aa} \\ (*\emph{Co-first authors with equal contribution})
  \item \bibentry{Lee:2023aa}
  \item \bibentry{Oh:2021wj}
  \item \bibentry{Lee:2020aa}
  \item \bibentry{Lee:2019ab}
  \item \bibentry{Lee:2019aa}
  \item \bibentry{Lee:2017aa}
  \item \bibentry{An:2017ui}
  \item \bibentry{Sohn:2016aa}
\end{itemize}

\textbf{Posters \& Technical Reports}\vspace{-4pt}
\begin{itemize}
  \item \bibentry{Lee:2021aa}
  \item \bibentry{Lee:2018aa}
  \item \bibentry{Lee:2017tn}
\end{itemize}

\textbf{Software Engineering Notes}\vspace{-4pt}
\begin{itemize}
  \item \bibentry{Langdon:2020vk}
\end{itemize}

\section{Academic Services}
\begin{itemize}
  
  \item Program committee: ISSTA'24, ASE'23 / (Artifact Evaluation Track) ECOOP'24, USENIX'24, ICSE'24, ISSTA'23, ICSME'22, ICSME'21
  \item Reviewer: TSE'24, IST'24, TOSEM'22, JSS'21, JSS'20  / (External) FSE'24, ECOOP'24, ICSE'23, ISSTA'23
\end{itemize}

\section{Invited Talks}
\begin{itemize}
  \item \emph{Statistical Program Analysis} \\ Korea Advanced Institute of Science and Technology (KAIST), 2024
  \item \emph{Statistical Program Analysis} \\ Ulsan National Institute of Science and Technology (UNIST), 2024
  \item \emph{Causal Program Dependence Analysis} \\ Sheffield Causality and Testing Workshop, 2023
  \item \emph{Statistical program dependence analysis} \\ Handong Global University, 2022
  \item \emph{Observation-based approximate dependency modeling and its use for program slicing} \\ Korea Conference on Software Engineering, 2022
  \item \emph{MOBS: Multi-Operator Observation-Based Slicing using Lexical Approximation of Program Dependence} \\ 59th CREST Open Workshop - Multi-language Software Analysis, 2018
\end{itemize}

\section{Grants and Fellowships}
\begin{itemize}
  \item Title: \emph{Statistical Security Analysis for Large, Evolving Software} \\ 
  Funding Agency: CASA - Cyber Security in the Age of Large-Scale Adversaries \\ 
  Grant ID: DFG under Germany's Excellence Strategy - \textbf{EXC 2092 CASA - 390781972} \\ 
  Amount: Salary according to the remuneration group E 14 TV-L (full time) \\
  Duration: 2024.01.01 -- 2025.12.31
\end{itemize}

\section{Awards and honors}
\begin{itemize}
  \item \textbf{PhD Dissertation Award}, School of Computing, KAIST, 2022
        \vspace{-5pt}\begin{itemize}
          \item \emph{Title of Dissertation: Statistical Program Dependence Approximation}
        \end{itemize}
  \item \textbf{2021 Naver Ph.D. Fellowship Award}: Awarded by NAVER Corp. to Ph.D. candidates who have published an outstanding research paper or have excellent publication performance, 2021
  \item Government-sponsored Scholarship, Ministry of Science and ICT of Korea, 2016 - 2022
  \item Government-sponsored Scholarship, Ministry of Science and ICT of Korea, 2012 - 2016
\end{itemize}


%-----------EXPERIENCE-----------
\section{Research Experience}
\resumeSubHeadingListStart

\resumeSubheading
{Software Security Group, MPI-SP}{Sep. 2022 -- Present}
{Postdoc}{Bochum, Germany}
\resumeItemListStart
\resumeItem{Researching on unbiased estimation of the missing mass/probability/expected number of discovering new classes of unknown multinomial distribution}
\resumeItem{Researched on extrapolating the coverage rate of the Greybox Fuzzing using the statistical model -- Greybox Fuzzing Extrapolation}
\resumeItem{Researched applying statisical methods for program analysis to overcome the scalability issue of the static analysis -- Statistical Reachability Analysis}
\resumeItemListEnd

\resumeSubheading
{Computational Intelligence for Software Engineering Laboratory (COINSE), KAIST}{Sep. 2016 -- Aug. 2022}
{Ph.D. Student}{Daejeon, Republic of Korea}
\resumeItemListStart
\resumeItem{Researched approximating the degree of dependence between program element using causal inference -- CPDA}
\resumeItem{Researched applying statistical models on the observation data to approximate the program dependence -- MOAD}
\resumeItem{Researched inferencing the type information in the binary executables using RNN with National Security Research Institute}
\resumeItem{Researched classifying the false positive alarms from static checker in continuous integration pipeline using CNN with Samsung Research}
\resumeItem{Researched program dependence approximation using the lexical model on the source code -- MOBS}
\resumeItemListEnd

\resumeSubheading
{Computational Intelligence for Software Engineering Laboratory (COINSE), KAIST}{Mar. 2016 -- Aug. 2016}
{Undergraduate Research Intern}{Daejeon, Republic of Korea}
\resumeItemListStart
\resumeItem{Researched on the amortised deep parameter optimisation of GPGPU work group size for OpenCV.}
\resumeItem{Accelerated the scalablility of Observation based slicing (ORBS) by applying a code distance metric during the slicing.}
\resumeItemListEnd

\resumeSubheading
{Programming Language Research Group (PLRG) Lab, KAIST}{Jul. 2015 -- Feb. 2016}
{Undergraduate Research Intern}{Daejeon, Republic of Korea}
\resumeItemListStart
\resumeItem{Developed a source code translator from C\# to C++ with F\#.}
\resumeItem{Developed a frontend of Scalable Analysis Framework for ECMAScript (SAFE), a Javascript static analysis tool.}
\resumeItemListEnd

\resumeSubHeadingListEnd
% \vspace{-16pt}




\section{Teaching Experience}
\begin{itemize}
  \item Teaching Assistant, Automated Software Testing (CS453), School of Computing, KAIST, Spring 2019
  \item Teaching Assistant, Artificial Intelligence Based Software Engineering (CS454), School of Computing, KAIST, Fall 2018
  \item Teaching Assistant, Introduction to Logic for Computer Science (CS402), School of Computing, KAIST, Spring 2018
  \item Teaching Assistant, Artificial Intelligence Based Software Engineering (CS454), School of Computing, KAIST, Fall 2017
  \item Teaching Assistant, Introduction to Logic for Computer Science (CS402), School of Computing, KAIST, Spring 2017
  \item Teaching Assistant, Special Topics in Computer Science $\langle$Search Based Software Engineering$\rangle$ (CS492), School of Computing, KAIST, Fall 2016

\end{itemize}


\end{document}