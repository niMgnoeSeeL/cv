\documentclass[letterpaper,11pt]{article}

\usepackage{latexsym}
\usepackage[empty]{fullpage}
\usepackage{titlesec}
\usepackage{marvosym}
\usepackage[usenames,dvipsnames]{color}
\usepackage{verbatim}
\usepackage{enumitem}
\usepackage{fancyhdr}
\usepackage{lastpage}
\usepackage[english]{babel}
\usepackage{tabularx}
\usepackage{fontawesome5}
\usepackage{multicol}
\usepackage[official]{eurosym}

\usepackage{bibentry}
\makeatletter\let\saved@bibitem\@bibitem\makeatother
\usepackage[hidelinks]{hyperref}
\makeatletter\let\@bibitem\saved@bibitem\makeatother

\setlength{\multicolsep}{-3.0pt}
\setlength{\columnsep}{-1pt}
\input{glyphtounicode}


\pagestyle{fancy}
\fancyhf{}
\fancyfoot{}
\fancyfootoffset[R]{0in}
\rfoot{\thepage/\pageref{LastPage}}
\renewcommand{\headrulewidth}{0pt}
\renewcommand{\footrulewidth}{0pt}

\addtolength{\oddsidemargin}{-0.6in}
\addtolength{\evensidemargin}{-0.5in}
\addtolength{\textwidth}{1.19in}
\addtolength{\topmargin}{-.4in}
\addtolength{\textheight}{.8in}

\urlstyle{same}

\raggedbottom
\raggedright
\setlength{\tabcolsep}{0in}

\titleformat{\section}{
  \vspace{-4pt}\scshape\raggedright\large\bfseries
}{}{0em}{}[\color{black}\titlerule \vspace{-5pt}]

\pdfgentounicode=1

\newcommand{\resumeItem}[1]{
  \item\small{
    {#1 \vspace{-2pt}}
  }
}

\newcommand{\classesList}[4]{
    \item\small{
        {#1 #2 #3 #4 \vspace{-2pt}}
  }
}

\newcommand{\resumeSubheading}[4]{
  \vspace{-2pt}\item
    \begin{tabular*}{1.0\textwidth}[t]{l@{\extracolsep{\fill}}r}
      \textbf{#1} & \textbf{\small #2} \\
      \textit{\small#3} & \textit{\small #4} \\
    \end{tabular*}\vspace{-7pt}
}

\newcommand{\resumeSubSubheading}[2]{
    \item
    \begin{tabular*}{0.97\textwidth}{l@{\extracolsep{\fill}}r}
      \textit{\small#1} & \textit{\small #2} \\
    \end{tabular*}\vspace{-7pt}
}

\newcommand{\resumeProjectHeading}[2]{
    \item
    \begin{tabular*}{1.001\textwidth}{l@{\extracolsep{\fill}}r}
      \small#1 & \textbf{\small #2}\\
    \end{tabular*}\vspace{-7pt}
}

\newcommand{\resumeSubItem}[1]{\resumeItem{#1}\vspace{-4pt}}

\renewcommand\labelitemi{$\vcenter{\hbox{\tiny$\bullet$}}$}
\renewcommand\labelitemii{$\vcenter{\hbox{\tiny$\bullet$}}$}

\newcommand{\resumeSubHeadingListStart}{\begin{itemize}[leftmargin=0.0in, label={}]}
\newcommand{\resumeSubHeadingListEnd}{\end{itemize}}
\newcommand{\resumeItemListStart}{\begin{itemize}}
\newcommand{\resumeItemListEnd}{\end{itemize}\vspace{-5pt}}



%-------------------------------------------
%%%%%%  RESUME STARTS HERE  %%%%%%%%%%%%%%%%%%%%%%%%%%%%


\begin{document}

\bibliographystyle{unsrt}
\nobibliography{mypublications}


{\Huge \scshape Seongmin Lee} \\ \vspace{10pt}
Max Planck Institute for Security and Privacy (MPI-SP) \\
Universitätsstraße 140 \\
44799 Bochum \\
Germany \\ \vspace{5pt}
\small \raisebox{-0.1\height}\faPhone\ +49 177 783 5480  ~ \href{mailto:seongmin.lee@mpi-sp.org}{\raisebox{-0.2\height}\faEnvelope\  \underline{seongmin.lee@mpi-sp.org}} ~
\href{https://scholar.google.com/citations?user=-YSnc6kAAAAJ&hl=en}{\raisebox{-0.2\height}\faUniversity\ \underline{Google Scholar}}  ~

\section{Research Summary}

The overarching objective of my research is to achieve \textbf{practical software testing in real-world scenarios} by addressing the empirical challenges associated with the \emph{scale and complexity of software systems}. To do so, I \textbf{utilize statistical methods}, such as \emph{causal inference, biostatistics, and machine learning}, \textbf{to analyze the dynamic behavior of software in operational environments}. My research has been published in top-tier software engineering venues, including ICSE, FSE, and JSS, and I served as a program committee member for top-tier conferences, including FSE, ASE, and ISSTA.

\section{Education and Employment}
\resumeSubHeadingListStart
\vspace{-2pt}
\item Max Planck Institute for Security and Privacy \hfill Germany \vspace{-2pt}
\vspace{-2pt} \item \hspace{10pt} \textbf{Postdoctoral Researcher, Software Security Research group} \hfill  Sep. 2022 -- Present
\vspace{-5pt} \item \hspace{20pt} Group head: Dr. Marcel Böhme

\vspace{-2pt}
\item Korea Advanced Institute of Science and Technology \hfill Republic of Korea \vspace{-2pt}
\vspace{-2pt} \item \hspace{10pt} \textbf{Doctor of Philosophy, School of Computing} \hfill  Sep. 2016 -- Aug. 2022
\vspace{-5pt} \item \hspace{20pt} Advisor: Dr. Shin Yoo

\vspace{-2pt} \item \hspace{10pt} \textbf{Bachelor of Science, School of Computing} \hfill  Feb. 2012 -- Aug. 2016
\vspace{-5pt} \item \hspace{10pt} \textbf{Bachelor of Science, Department of Mathematical Sciences}
\resumeSubHeadingListEnd

\section{Publications}
\textbf{Refereed Journal Articles}\vspace{-4pt}
\begin{itemize}[leftmargin=2cm]
  \item[SCP'25] \bibentry{leeCausalProgramDependence2025a}
  \item[JSS'21] \bibentry{leeObservationbasedApproximateDependency2021}
  \item[JSS'20] \bibentry{leeEvaluatingLexicalApproximation2020}
\end{itemize}

\textbf{Refereed Conference Publications}\vspace{-4pt}
\begin{itemize}[leftmargin=2cm]
  \item[ICSE'25] \bibentry{leeAccountingMissingEvents2025}
  \item[ICSE'24] \bibentry{liyanageExtrapolatingCoverageRate2024}  \\ (*\emph{Co-first authors with equal contribution})
  \item[FSE'23] \bibentry{leeStatisticalReachabilityAnalysis2023}
  \item[SCAM'19] \bibentry{leeMOADModelingObservationBased2019}
  \item[ICST'19] \bibentry{leeClassifyingFalsePositive2019}
\end{itemize}

% \newpage

\textbf{Preprints}\vspace{-4pt}
\begin{itemize}[leftmargin=2cm]
  % \item[] \bibentry{leeStructureawareResidualRisk2025}
  \item[] \bibentry{liuCanLLMGenerate2025}\\ (*\emph{Co-first authors with equal contribution})
  \item[] \bibentry{leeHowMuchUnseen2024}
\end{itemize}

\textbf{Invited Articles}\vspace{-4pt}
\begin{itemize}[leftmargin=2cm]
  \item[GI'20] \bibentry{langdonGeneticImprovementICSE2020}
\end{itemize}

\textbf{Refereed Workshop Publications}\vspace{-4pt}
\begin{itemize}[leftmargin=2cm]
  \item[ICST'21] \bibentry{ohEffectivelySamplingHigher2021}
  \item[ICSE'20] \bibentry{leeScalableApproximateProgram2020}
  \item[KCC'19] \bibentry{anPYGGIPythonGeneral2017}
  \item[ICSE'18] \bibentry{leeMOBSMultiOperatorObservationBased2018}
  \item[SBSE'17] \bibentry{leeHyperheuristicObservationBased2017}
  \item[SBSE'16] \bibentry{sohnAmortisedDeepParameter2016}
\end{itemize}

\section{Grants and Fellowships}
\begin{itemize}
  \item Title: \emph{Statistical Security Analysis for Large, Evolving Software} \\
        Funding Agency: CASA - Cyber Security in the Age of Large-Scale Adversaries \\
        Grant ID: DFG under Germany's Excellence Strategy - \textbf{EXC 2092 CASA - 390781972} \\
        Amount: Salary according to the remuneration group E 14 TV-L (full time, $\sim$ \euro{136,000}) \\
        Duration: 2024.01.01 -- 2025.12.31
\end{itemize}

\section{Awards and Honors}
\begin{itemize}
  \item \textbf{Distinguished Artifact Reviewer Award}, 33rd USENIX Security Symposium, 2024
  \item \textbf{PhD Dissertation Award}, School of Computing, KAIST, 2022
        \vspace{-5pt}\begin{itemize}
          \item \emph{Title of Dissertation: Statistical Program Dependence Approximation}
        \end{itemize}
  \item \textbf{2021 Naver Ph.D. Fellowship Award}: Awarded by NAVER Corp. to Ph.D. candidates who have published an outstanding research paper or have excellent publication performance, 2021
  \item Government-sponsored Scholarship, Ministry of Science and ICT of Korea, 2016 - 2022
  \item Government-sponsored Scholarship, Ministry of Science and ICT of Korea, 2012 - 2016
\end{itemize}

\section{Services}
\textbf{Academic Services}\vspace{-4pt}
\begin{itemize}
  \item Program committee: (Main Track) ASE'24, ISSTA'24, FUZZING'24, SCAM'24, ASE'23 / (Artifact Evaluation Track) ISSTA'24, ECOOP'24, USENIX Security'24, ICSE'24, ISSTA'23, ICSME'22, ICSME'21 / (Student Research Competition Track) FSE'24 / (Tool Demonstration Track) ASE'24
  \item Reviewer: TOSEM'24, TSE'24, IST'24, ASE'24, TOSEM'22, JSS'21, JSS'20  / (External) ICSE'24, FSE'24, ECOOP'24, ICSE'23, ISSTA'23
\end{itemize}

\textbf{Institutional Services}\vspace{-4pt}
\begin{itemize}
  \item Open Science Ambassador, Max Planck Institute for Security and Privacy \hfill   2023 -- Present  \vspace{-6pt}
  \begin{itemize}[leftmargin=.5cm]
    \item[] As representatives of the institute, Open Science Ambassadors raise awareness and provide valuable information to their colleagues about Open Science practices. They meet annually, in person or online, to discuss strategies, participate in workshops, and evaluate the status of Open Science within and beyond the Society.
  \end{itemize}
\end{itemize}


\section{Advising Experience}
\textbf{Ph.D. Students}\vspace{-4pt}
\begin{itemize}
  \item \textbf{Danushka Liyanage} (Monash University, moved on to University of Sydney Postdoc, 2024) \hfill Nov. 2022 -- Dec. 2023 \vspace{-6pt}
  \begin{itemize}[leftmargin=.5cm]
    \item[] Co-advised with Dr. Marcel Böhme on extrapolating the coverage rate of the Greybox Fuzzing using the statistical model. A \textbf{full conference paper} where Danushka is the first author has been accepted to ICSE'24.
  \end{itemize}
\end{itemize}

\textbf{Undergraduate Students}\vspace{-4pt}
\begin{itemize}
  \item \textbf{Jing Liu} (Shanghai University, incoming Ph.D. student in UC Irvine) \hfill Apr. 2024 -- Present \vspace{-6pt}
  \begin{itemize}[leftmargin=.5cm]
    \item[] Co-advised with Dr. Marcel Böhme on the LLM-based regression test generation for the software testing. A \textbf{full conference paper} where Jing is the first author has been submitted to FSE'25.
  \end{itemize}
  % \newpage
  \item \textbf{Shreyas Minocha} (Rice University, moved on to Georgia Tech Ph.D. student, 2024) \hfill Feb. 2023 -- Jul. 2024 \vspace{-6pt}
  \begin{itemize}[leftmargin=.5cm]
    \item[] Primary advisor on the statistical information leakage analysis to suggest the accurate and safe information leakage estimator. A \textbf{full conference paper} has been accepted to ICSE'25.
  \end{itemize}
  \item \textbf{Saeyoon Oh} (KAIST, moved on to FuriosaAI) \hfill Jul. 2020 -- Apr. 2021 \vspace{-6pt}
  \begin{itemize}[leftmargin=.5cm]
    \item[] Co-advised with Dr. Shin Yoo on the effective sampling of higher-order mutants for mutation testing. A \textbf{workshop paper} where Saeyoon is the first author has been accepted to ICST'21.
  \end{itemize}
\end{itemize}


\section{Teaching Experience}

\textbf{Guest Lecturer}\vspace{-4pt}
\begin{itemize}
  \item \emph{Guarantees in Software Security -- 2nd part: Extrapolating Software Testing}, \\Fuzzing and Software Security Summer School @ National University of Singapore (NUS) \hfill May 2024
\end{itemize}

\textbf{Invited Talks}\vspace{-4pt}
\begin{itemize}
  \item \emph{Statistical Program Analysis}, Korea Advanced Institute of Science and Technology (KAIST) \hfill Jan. 2024
  \item \emph{Statistical Program Analysis}, Ulsan National Institute of Science and Technology (UNIST) \hfill Jan. 2024
  \item \emph{Causal Program Dependence Analysis}, Sheffield Causality and Testing Workshop \hfill Sep. 2023
  \item \emph{Statistical program dependence analysis}, Handong Global University \hfill Aug. 2022
  \item \emph{Observation-based approximate dependency modeling and its use for program slicing}, \\Korea Conference on Software Engineering \hfill Jan. 2022
  \item \emph{MOBS: Multi-Operator Observation-Based Slicing using Lexical Approximation of Program Dependence}, \\59th CREST Open Workshop -- Multi-language Software Analysis \hfill Mar. 2018
\end{itemize}

\textbf{Teaching Assistant}\vspace{-4pt}
\begin{itemize}
  \item Automated Software Testing (CS453), School of Computing (SoC), KAIST \hfill Spring 2019
  \item Artificial Intelligence Based Software Engineering (CS454), SoC, KAIST \hfill Fall 2018
  \item Introduction to Logic for Computer Science (CS402), SoC, KAIST \hfill Spring 2018
  \item Artificial Intelligence Based Software Engineering (CS454), SoC, KAIST \hfill Fall 2017
  \item Introduction to Logic for Computer Science (CS402), SoC, KAIST \hfill Spring 2017
  \item Special Topics in Computer Science $\langle$Search Based Software Engineering$\rangle$ (CS492), SoC, KAIST \hfill Fall 2016

\end{itemize}


\section{Research Experience}
\resumeSubHeadingListStart

\resumeSubheading
{Software Security Group, MPI-SP}{Sep. 2022 -- Present}
{Postdoc}{Bochum, Germany}
\resumeItemListStart
\resumeItem{Researching on unbiased estimation of the missing mass/probability/expected number of discovering new classes of unknown multinomial distribution}
\resumeItem{Researched on LLM-based regression test generation for the software security testing -- Cleverest}
\resumeItem{Researched applying biostatistics for information leakage analysis to suggest the accurate and safe information leakage estimator -- Statistical Information Leakage Analysis}
\resumeItem{Researched on extrapolating the coverage rate of the Greybox Fuzzing using the statistical model -- Greybox Fuzzing Extrapolation}
\resumeItem{Researched applying statisical methods for program analysis to overcome the scalability issue of the static analysis -- Statistical Reachability Analysis}
\resumeItemListEnd

\resumeSubheading
{Computational Intelligence for Software Engineering Laboratory (COINSE), KAIST}{Sep. 2016 -- Aug. 2022}
{Ph.D. Student}{Daejeon, Republic of Korea}
\resumeItemListStart
\resumeItem{Researched approximating the degree of dependence between program element using causal inference -- CPDA}
\resumeItem{Researched applying statistical models on the observation data to approximate the program dependence -- MOAD}
\resumeItem{Researched inferencing the type information in the binary executables using RNN with National Security Research Institute}
\resumeItem{Researched classifying the false positive alarms from static checker in continuous integration pipeline using CNN with Samsung Research}
\resumeItem{Researched program dependence approximation using the lexical model on the source code -- MOBS}
\resumeItemListEnd

\resumeSubheading
{Computational Intelligence for Software Engineering Laboratory (COINSE), KAIST}{Mar. 2016 -- Aug. 2016}
{Undergraduate Research Intern}{Daejeon, Republic of Korea}
\resumeItemListStart
\resumeItem{Researched on the amortised deep parameter optimisation of GPGPU work group size for OpenCV.}
\resumeItem{Accelerated the scalablility of Observation based slicing (ORBS) by applying a code distance metric during the slicing.}
\resumeItemListEnd

\resumeSubheading
{Programming Language Research Group (PLRG) Lab, KAIST}{Jul. 2015 -- Feb. 2016}
{Undergraduate Research Intern}{Daejeon, Republic of Korea}
\resumeItemListStart
\resumeItem{Developed a source code translator from C\# to C++ with F\#.}
\resumeItem{Developed a frontend of Scalable Analysis Framework for ECMAScript (SAFE), a Javascript static analysis tool.}
\resumeItemListEnd

\resumeSubHeadingListEnd

\end{document}