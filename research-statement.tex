\documentclass{article}
\usepackage[margin=1in]{geometry} % Optional: Adjust page margins
\usepackage{fancyhdr} % For custom headers
\usepackage{booktabs}
\usepackage{hyperref}
\usepackage{xcolor}
\usepackage{enumitem}
\usepackage{kotex}
\usepackage{titlesec}

% make reference blue
\hypersetup{
  colorlinks,
  linkcolor={blue},
  citecolor={blue},
  urlcolor={blue}
}
\renewcommand{\normalsize}{\fontsize{9pt}{11pt}\selectfont}
\titleformat{\section}
  {\normalfont\fontsize{11}{0}\bfseries}{\thesection}{1em}{}

% Define the global footer
\pagestyle{fancy}
\fancyhf{} % Clear all header/footer fields
\fancyfoot[R]{Page \thepage} % Footer centered on all pages


% Define the header for the first page
\fancypagestyle{firstpage}{
    \fancyhf{} % Clear all header/footer fields
    \fancyhead[L]{\textbf{Research Statement}} % Left-aligned header
    % \fancyhead[C]{Center Header (optional)} % Centered header
    \fancyhead[R]{\textbf{Seongmin Lee}} % Right-aligned header
    \fancyfoot[R]{Page \thepage} % Footer right
}

\begin{document}

\thispagestyle{firstpage} % Apply the custom header only to the first page


\noindent I specialize in software engineering, with a focus on program analysis and software testing. The overarching objective of my research is to \textbf{\color{blue} make program analysis scalable} by addressing the challenges associated with the scale and complexity of software systems and to \textbf{\color{blue} achieve practical software testing in real-world scenarios}. I \textbf{utilize statistical methods to analyze dynamic information from program execution}, reasoning about programs' semantic behavior in operational environments and tackling empirical challenges in software testing.

Over the past few decades, the widespread adoption of program analysis and software testing has become integral to ensuring the reliability and security of software applications. Conventional program analysis relies on \emph{formal semantics}, which assigns rigorous mathematical meaning to the syntax of a programming language, to deduce a program's semantic features. However, formal semantics are limited in their ability to handle the heterogeneous features prevalent in modern software, such as network communication, system-level behavior, and third-party libraries, which are often beyond what formal semantics cover.

% Software testing, finding defects in the software by actively executing it, got notable attention as formally proving the correctness of the software is often unfeasible due to the gigantic space of the program state. However, its practicality is challenged by its unpredictable and undersampling nature;  there is always an unseen behavior in the software, and whether there is an undetected defect and if the testing process will find it, given the limited number of test cases, is unknown.

Software testing, which finds defects by actively executing the software, has gained notable attention since formally proving correctness is often unfeasible due to the vast program state. However, its practicality is challenged by its unpredictability and undersampling nature; there is always an unseen behavior in the software, and whether there is an undetected defect and if the testing process will find it, given the limited number of test cases, is unknown.

My research addresses the inherent limitations of program analysis and software testing fundamentally by reframing these tasks as \emph{statistical problems} and solve them by employing the statistical methods to the dynamic information from the program execution. The primary advantage of \emph{inferring} the program semantics statistically for program analysis is that they \emph{operate irrespective of the system's complexity}, even in cases where the entire system is unknown, inaccessible, and/or undecidable. Statistical inference for software testing, focusing on the operational distribution of the program executions, provides \emph{predictions} and \emph{guarantees} for the software testing process in practice.

% I have been dedicated to addressing various research challenges within the field of software engineering, targeting issues that are either inherently difficult to scale or have remained unexplored due to the limitations of conventional program analysis/software testing. The statistical methods under consideration encompass a broad spectrum including causal inference, bio-statistics, machine learning, etc., from various domains, such as ecology, linguistics, etc., and are specifically customized for the intricacies of the software engineering domain. The following outlines the specific research topics on which I have focused my efforts:

In addressing the aforementioned limitations, I leverage a diverse range of statistical methods, including causal inference, biostatistics, and machine learning, drawn from fields such as ecology, linguistics, and social sciences. These methods are not only adapted but also customized to address the intricacies of software engineering. The following outlines the specific research topics on which I have focused my efforts:

\begin{itemize}[leftmargin=*]
  \item \textbf{Counterfactual program analysis~\cite[Section~\ref{sec:dependency}]{leeObservationbasedApproximateDependency2021,leeCausalProgramDependence2025a,leeEvaluatingLexicalApproximation2020,ohEffectivelySamplingHigher2021}}:
        I have developed statistical methodologies to infer program dependencies, determining which program elements affect others, by considering counterfactual events in sample program executions. Our work demonstrates that these statistical methods can identify dependencies not recognized by conventional program analysis.
  \item \textbf{Reasoning the unseen in software testing~\cite[Section~\ref{sec:unseen}]{leeStatisticalReachabilityAnalysis2023,leeStructureawareResidualRisk2025,leeAccountingMissingEvents2025,liyanageExtrapolatingCoverageRate2024}}:
        I have developed statistical methodologies to estimate the likelihood of unseen events in software testing, providing reliable interpretations of testing results and predicting the future performance of the testing process. These methodologies outperform existing models based on formal semantics and statistical approaches that mishandle the unseen in practical scenarios.
  \item \textbf{Refining statistical methods for software engineering~\cite[Section~\ref{sec:refine}]{leeCausalProgramDependence2025a,leeStatisticalReachabilityAnalysis2023,leeHowMuchUnseen2024,leeStructureawareResidualRisk2025}}: 
        Statistical models from diverse fields, such as social science and ecology, offer valuable insights. However, the software domain introduces distinct characteristics, including discreteness, determinism, and structural dependencies. Recognizing these unique attributes, we develop specialized statistical methodologies to address the specific challenges of software engineering, aiming to enhance the applicability of statistical methods in this domain.
\end{itemize}

% The rest of the research statement will elaborate on the research topics I have been working on and the future research directions I plan to pursue.



\section{Counterfactual Program Analysis}
\label{sec:dependency}

At the heart of program analysis is understanding dynamic program operation, which arises from interactions among elements like functions, basic blocks, and variables—i.e., \emph{the dependency relations between program elements}. My research establishes a novel dependency analysis paradigm, \emph{observation-based dependency analysis}~\cite{leeEvaluatingLexicalApproximation2020,leeObservationbasedApproximateDependency2021,leeCausalProgramDependence2025a}, using statistical methods to infer dependencies between program elements without relying on formal semantics. The statistical methods I have developed are specifically tailored for \emph{counterfactual reasoning}, inferring dependencies between program elements by observing the effects of manipulating one program element's value on another. For instance, if changing the value of program element $B$ results in a change in the value of program element $A$, we infer that program element $A$ depends on program element $B$.

Observation-based dependency analysis addresses the limitations of traditional formal semantics-based methods, namely: 1) infeasibility for features beyond the coverage of formal semantics, 2) false positives from over-approximation, and 3) the undecidable nature of program analysis, which hampers scalability and practicality. The statistical methods I have developed are effective regardless of the system’s \emph{heterogeneity} or even when parts of the system are unknown or inaccessible. These methods are \emph{empirically validated}, \emph{data-driven}, and provide a \emph{quantifiable}, statistically grounded approximation of dependencies for the inherently undecidable problem of dependence.

% At the heart of program analysis is the comprehension of dynamic program operation, arising from interactions among elements like functions, basic blocks, and variables. \emph{Identifying dependencies between these elements} is crucial for understanding program operation and uncovering issues such as bugs. Traditionally, static analysis is employed to identify program dependencies by deriving them from the formal semantics of the programming language. However, static analysis faces significant limitations, namely, 1) infeasibility for features beyond formal semantics coverage, 2) false positives due to over-approximation, and the 3) inherently undecidable nature of the program, hampering its scalability and practicality.

% My research focuses on establishing a novel dependency analysis paradigm using statistical methods inferring dependencies between program elements without relying on formal semantics.
% The statistical methods I have developed are specifically tailored for \emph{counterfactual reasoning}. The aim is to infer dependencies between program elements without relying on formal semantics.
% The core concept involves observing dependencies during intervened program execution, where the manipulation of one program element's value affects another. For instance, if manipulating the value of program element $B$ results in a change in the value of program element $A$, we infer that program element $A$ depends on program element $B$.

% Such an approach inherently overcomes the limitations of static analysis.
% Statistical methods perform effectively regardless of the system's \emph{heterogeneity} or even when the entire/partial system is unknown or inaccessible. They are \emph{empirically evidenced} and \emph{data-driven}.
% In addition, it provides a \emph{quantifiable} and statistically grounded approximation of the dependencies for the undecidable problem of dependence.

\vspace{0.5em} \noindent
Following are the research works I have conducted in this direction.

\vspace{0.5em}
\noindent\textbf{Identifying the dependency.}
In our prior work, MOAD~\cite{leeObservationbasedApproximateDependency2021} demonstrated the successful approximation of program element dependencies through statistical methods. It reframes program dependency as the likelihood that one program element affects another. We devised an efficient sampling strategy to capture changed program elements during intervened program execution and applied statistical methods to estimate dependency likelihood.
% 
Our evaluation revealed that, for programs analyzable by conventional methods, MOAD achieved high accuracy in identifying the set of program elements influencing the target element compared to the traditional program analysis. For programs beyond conventional analysis capabilities, MOAD uncovered hidden dependencies not identified by traditional methods.


\vspace{0.5em}
\noindent\textbf{Quantifying the strength of the dependency.}
Not all dependencies are equal; A variable's value may exhibit sensitivity to changes in some variables (strong dependency), be rarely affected by others (weak dependency), or remain unresponsive to changes in the rest (no dependency). Reframing the program dependency as a sensitivity to changes can overcome the limitations of the undecidable nature of the program semantics and provide a nuanced understanding of the program operation. In our work, CPDA~\cite{leeCausalProgramDependence2025a}, we utilized causal inference to quantify the dependency strength solely from dynamic information. Our empirical findings revealed that fine-grained dependency information can group program elements into functional clusters, enhancing debugging productivity. Our Subsequent research~\cite{ohEffectivelySamplingHigher2021} demonstrated the effectiveness of this information in software testing. Mutating pairs of program elements with strong dependencies are more likely to generate higher-order mutants that are challenging to detect with existing test cases.

% \vspace{0.5em}
% Furthermore, we explore additional sources of information that statistical methods can leverage to infer program operation. In our work, MOBS~\cite{leeEvaluatingLexicalApproximation2020}, we specifically considered the lexical information of program elements, such as variable and function names, assuming that related functionalities share lexically similar names. To enhance dynamic analysis efficiency, we developed a lexical similarity model and integrated it into the dynamic analysis.

\section{Reasoning the Unseen in Software Testing}
\label{sec:unseen}

Software testing is fundamentally a \emph{sampling process}, where test cases from the operational distribution—i.e., the distribution of program executions in practice—are executed to uncover defects. Given the vast space of possible test cases, exhaustively covering all program behaviors is infeasible. Consequently, software testing is inherently vulnerable to unseen behaviors, leading to two concrete challenges: 1) the interpretation of test results becomes unreliable, and 2) there is a never-ending concern about the sufficiency of the testing process.

Another research direction I have pursued focuses on developing a statistical inference model to address unseen behaviors in software testing. By quantifying the likelihood of unseen behaviors and incorporating it into test results, the model provides a \emph{reliable interpretation of testing outcomes}~\cite{leeStatisticalReachabilityAnalysis2023,leeAccountingMissingEvents2025}. Additionally, it predicts the future performance of the testing process with statistical guarantees, enabling a rational assessment of testing effectiveness and serving as a decision-maker for resource allocation~\cite{liyanageExtrapolatingCoverageRate2024,leeStructureawareResidualRisk2025}. These advancements enhance the \emph{practicality of software testing}.


\vspace{0.5em} \noindent
The following are the research works I have conducted in this direction.

\vspace{0.5em}
\noindent\textbf{Extrapolating the greybox fuzzing.}
Fuzzing, an automated software testing technique generating numerous test cases, is one of the industry's most widely adopted methods. Yet, little is known about how to determine the effectiveness of fuzzing, whether it will uncover new defects, or what its future performance will be. While blackbox fuzzing, with its consistent sampling distribution, is relatively predictable, greybox fuzzing introduces complexities due to adaptive bias, where the sampling distribution evolves based on test input execution coverage. In our work~\cite{liyanageExtrapolatingCoverageRate2024}, we introduced a novel statistical model for predicting greybox fuzzing performance. Leveraging ecological statistics, our model forecasts future coverage increases in the stochastic process. To address adaptive bias, we partition the coverage record into sub-records, apply ecological statistics to each, and regress predicted coverage increases for extrapolating future performance. Our evaluation shows that this adaptive bias-aware model outperforms existing approaches, offering improved predictions for greybox fuzzing performance.

\vspace{0.5em}
\noindent\textbf{Reliable information leakage analysis.} Information leakage analysis quantifies the information leaked from a secret source to a public sink during program execution. Existing statistical methods rely on mutual information estimation, which is sensitive to missing observations, often leading to either a significant bias or a false sense of security. In our work~\cite{leeAccountingMissingEvents2025}, we developed a novel mutual information estimator that addresses missing observations, enabling accurate and secure leakage estimation even with limited data. Our evaluation shows that our estimator outperforms existing methods in both accuracy and security, enhancing the reliability of information leakage analysis.

\vspace{0.5em}
\noindent\textbf{Estimating the reaching probability.} Quantitative reachability analysis measures the probability of reaching a specific program state, such as an errornous state or a defect-inducing state, during execution. While conventional methods compute this probability through symbolic execution and model counting, we developed the statistical reachability analysis~\cite{leeStatisticalReachabilityAnalysis2023} that estimates the reaching probability from the sample program executions. In small-scale programs with known operational distribution, our approach outperformed conventional methods in both accuracy, due to the limited coverage of the formal semantics, and time cost. Our method also demonstrated its effectiveness in large-scale real-world software with an unknown operational distribution, as encountered in software fuzzing.

\section{Refining Statistical Methods for Software Engineering}
\label{sec:refine}

The software domain has unique characteristics that set it apart from fields where statistical methods are commonly applied, such as ecology, linguistics, and social sciences. Unlike these nature-based environments, the software domain features a discrete and deterministic nature, structural dependencies, and unnatural distributions. Recognizing these distinct traits, we refine statistical methods to improve their applicability in software engineering.

\vspace{0.5em}
\noindent\textbf{Program-specific characteristics.}
Programs exhibit a highly structured nature, with dependencies between program elements at the core of their operation. When applying statistical methods to software engineering tasks, we account for this structural aspect, refining the methods for more accurate estimations. For instance, in our reachability analysis~\cite{leeStatisticalReachabilityAnalysis2023}, we proposed a structure-aware reaching probability estimator. Unlike existing statistical estimators (e.g., Laplace smoothing), which treat all unreached program elements equally, our estimator distinguishes them based on structural dependencies, assigning probabilities accordingly. Similarly, in residual risk analysis~\cite{leeStructureawareResidualRisk2025}, we recognized the importance of structural dependency. Residual risk—the risk of a defect remaining after testing—is typically upper-bounded by the discovery probability of uncovered program elements. While existing methods assume independent coverage events, our structure-aware analysis incorporates structural dependencies, i.e., control-flow, providing a significantly tighter upper bound on residual risk.

% our work~\cite{leeStatisticalReachabilityAnalysis2023}, we introduced structure-aware statistical reachability analysis. This approach, leveraging sample executions, estimates the reaching probability of program elements. Unlike existing probability estimators, such as Laplace smoothing or Good-Turing estimation, which treat all unreached program elements equally, our structure-aware estimator assigns different probabilities to unreached elements based on structural dependencies. This refinement provides a more accurate estimation of reaching probability.


% We take account of this dependency relationship, refining the statistical methods to 

% We delve into this dependency relationship, refining 

% we introduced structure-aware statistical reachability analysis~\cite{leeStatisticalReachabilityAnalysis2023}. This approach, leveraging sample executions, estimates the reaching probability of program elements. Unlike existing probability estimators, such as Laplace smoothing or Good-Turing estimation, which treat all unreached program elements equally, our structure-aware estimator assigns different probabilities to unreached elements based on structural dependencies. This refinement provides a more accurate estimation of reaching probability.

Recognizing the deterministic nature of software enables more accurate identification of dependencies between program elements. In causal analysis, structure discovery identifies causal relationships from observational data. While existing methods are designed for probabilistic systems, we introduced a novel structure discovery method~\cite{leeCausalProgramDependence2025a} tailored to deterministic dependencies in software. Specifically, if the value of program element $A$ changes when $B$ is manipulated, $A$ is unequivocally dependent on $B$. Our method significantly improves causal structure accuracy for program dependency analysis compared to existing approaches.

% \vspace{0.5em}
% \noindent\textbf{Absence of observations.}
% The absence of observations is a common challenge in empirical science. In the software domain, it is more prevalent due to the extensive cardinality of the program state space. For instance, information leakage analysis utilizes mutual information between secret and public variables to quantify information leakage. However, the large domain size of secret/public values often results in numerous missing pairs during sample execution. We demonstrated that these missing pairs can introduce significant bias or provide a false sense of security in existing mutual information estimation. To address this issue, we introduced a novel mutual information estimator~\cite{chaomi2024} that accounts for missing observations. This estimator provides accurate and secure mutual information estimation, even with limited observations.

\vspace{0.5em}
\noindent\textbf{Distribution-specific estimation.}
Many statistical methods propose a single general estimator for the unknown quantity of the underlying distribution, irrespective of its shape. While some well-known distributions, like Zipf's law, are common in natural environments, the software domain often exhibits heterogeneous and unnatural distributions. In such cases, a distribution-specific estimator may outperform the general estimator. Our recent work~\cite{leeHowMuchUnseen2024} contributes to a purely statistical domain, focusing on estimating the missing probability mass of a distribution. Combining statistical theory with optimization methods, specifically genetic algorithms, we introduce a 'distribution-free methodology' to discover a 'distribution-specific estimator' that surpasses the general estimator in performance.


\section{Future Research Directions}


% \subsubsection*{Long-term Vision}
% \vspace{0.5em}
\noindent\textbf{Long-term Vision}
\vspace{0.5em}

\noindent The ultimate goal of my research is to make program analysis scalable and applicable in modern software development environments. To achieve this, I have integrated advanced statistical methods with dynamic program information, enabling scalability for the heterogeneity of modern software and improving software testing performance. \textbf{I believe this is just the beginning}: Modern software development generate vast amounts of data, and leveraging this data—particularly through machine learning—has become standard practice. However, other statistical methods, which offer complementary benefits, have received less attention. For example, \emph{unseen event estimation}~\cite{leeStatisticalReachabilityAnalysis2023,leeHowMuchUnseen2024,leeStructureawareResidualRisk2025,leeAccountingMissingEvents2025,liyanageExtrapolatingCoverageRate2024} addresses data scarcity, and \emph{counterfactual causal analysis}~\cite{leeObservationbasedApproximateDependency2021,leeCausalProgramDependence2025a,leeEvaluatingLexicalApproximation2020,ohEffectivelySamplingHigher2021} provides explanations from data—both critical for software development where machine learning often falls short. Beyond these, statistical methods such as \emph{Bayesian modeling}, \emph{extreme value theory}, and \emph{robust inference} can bring unique benefits to software engineering. In the long term, I aim to adapt and integrate these approaches into software engineering to improve scalability and practicality.

% I address limitations of conventional program analysis by incorporating statistical methods into dynamic information, enabling scalability to the heterogeneity of modern software and enhancing software testing performance.
% The two primary statistical methodologies I employed are \emph{counterfactual causal analysis}~\cite{leeObservationbasedApproximateDependency2021,leeCausalProgramDependence2025a,leeEvaluatingLexicalApproximation2020,ohEffectivelySamplingHigher2021} and \emph{unseen event estimation}~\cite{leeStatisticalReachabilityAnalysis2023,leeHowMuchUnseen2024,leeStructureawareResidualRisk2025,leeAccountingMissingEvents2025,liyanageExtrapolatingCoverageRate2024}. Still, various statistical methods that are influential in other domains can be integrated into the software engineering domain, including, but not limited to, \emph{Bayesian modeling}, \emph{Extreme value theory}, \emph{Overparamterized model}, \emph{Robust inference}, etc. In a long-term perspective, I aim to adjust and integrate these statistical methods to the software engineering domain to improve the scalability and practicality.

% \subsubsection*{Short-term Goals: Next Five Years Plan}
\vspace{0.5em}
\noindent\textbf{Short-term Goals: Next Five Years Plan}
\vspace{0.5em}

\noindent 

\noindent Despite its success, statistical program analysis faces distinct challenges that are fundamentally different from the conventional program analysis. Those challenges includes, but not limited to, 
\begin{itemize}[leftmargin=20pt, topsep=2pt, parsep=0pt, partopsep=0pt, label={-}]
  \item[\textbf{C1.}] \emph{Making sufficiently precise statements about properties of rarely executed components}: Most of the hard-to-find bugs lie in the rarely executed program components, and missing the behavior of those components can lead to a misprediction of the analysis result, which can be critical in various safety-critical scenarios.
  \item[\textbf{C2.}] \emph{Efficiently adjusting statistical program analysis in the presence of program evolution}: Re-executing the updated program is required in order to generate the new analysis result, which is time-consuming.
  \item[\textbf{C3.}] \emph{Adapting the statistical reasoning to the different domain/distribution}: For every empirical research, the domain shift is a critical issue. It occurs when the distribution of the observational data is different from the distribution of the data that the model is trained on, which can lead to a significant bias in the analysis result.
\end{itemize}
In the short term, I aim to address these challenges through specific research plans, some already in progress.

\vspace{0.5em}
\noindent\textbf{Modularized statistical program analysis (C1,2,3).}
I propose a new statistical program analysis that is \emph{modularized} and \emph{compositional} by design. The methodology involves generating observational data for each software component, inferring its statistical reachability model, and composing these models to analyze the entire system. By focusing on individual components, ample data is collected even for rarely observed ones, contributing to the final analysis result. The modularized approach simplifies adaptation to software changes; only updated components require re-analysis, enabling reuse of prior analysis. Additionally, observation and inference for each component can be parallelized, significantly reducing analysis time.

The main challenge is composing statistical models for each component to derive the final analysis for the entire software system. This involves adapting analysis from the independent domain of each component to the context-aware domain of whole program execution—a topic explored in the next research direction. This ongoing research recently secured funding from CASA--{\footnotesize Cyber Security in the Age of Large-Scale Adversaries}, with me as the sole PI.

\vspace{0.5em}
\noindent\textbf{Adapting to the different domain (C3).}
The domain shift is a prevalent issue in the software testing due to the varied execution environments and differences between in-house and production settings. Previously the machine-learning community has developed the domain adaptation method to address the domain shift, which I aim to adapt to the software domain. Two potential approaches are to be explored: 1) composing non-parametric models from each component using a covariate shifting method (e.g., importance sampling) and 2) designing parametric models for each component for the Markov chain model of the reachability.

I propose collaboration with industry partners, highlighting the paramount importance of this issue in practical applications. Many bugs occur in the software, even though it is tested during the development, due to the discrepancy between the development and production environment, and it could cause a significant loss in the industry.
The industry's vested interest in overcoming domain shift challenges makes this research particularly appealing, as it aligns with their pressing concerns. Through this collaboration, we aim to validate the effectiveness of proposed solutions in real-world settings, offering practical insights to mitigate domain shift challenges within the industry.


\vspace{0.5em}
\noindent\textbf{Integration of static and statistical program analysis (C1).}
Static and dynamic analyses complement each other in program analysis. Static analysis scales effectively in systems with full semantic coverage and provides formal guarantees, while dynamic analysis addresses unknown or inaccessible systems through execution observation with statistical guarantees. Static analysis excels at capturing rare behaviors triggered by specific conditions (e.g., \texttt{if (val == 42)}), which are difficult to observe in arbitrary executions. I will explore integrating static and statistical analyses to combine their strengths: using statistical methods to infer the behavior of components not analyzable by static methods, expanding the analysis domain while maintaining both formal and statistical guarantees. The main challenge is identifying suitable components for each method and integrating their guarantees.

{\small
\bibliographystyle{acm}
\bibliography{ref}
}
\end{document}